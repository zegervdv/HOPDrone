\documentclass[a4paper]{article}        % Standaard een kolom layout
\usepackage[english]{babel}             % Stel woordafbrekingen en referentienamen in
\usepackage{graphicx}                   % Afbeeldingen weergeven
\usepackage{float}                      % Figuren op plaats waar ze gedefinieerd staan: [H]
\usepackage{helvet}                     % Lettertype instellen op Helvetica
\usepackage[hidelinks]{hyperref}        % Referenties aanklikbaar in PDF, geen kaders rond weergeven
\usepackage{siunitx}                    % SI unit symbolen
\usepackage{amsmath}                    % Matrices en vergelijkingen
\usepackage{subcaption}                 % Subfiguren
\usepackage[parfill]{parskip}			% Niet inspringen aan begin alinea


\title{Hardware Design Project\\ Designing a drone positioning sensor}
\author{Laurens Bogaert\\Thomas Deckmyn\\Zeger Van de Vannet}
\date{\today}

\begin{document}
\maketitle

% Inhoudstafel
\newpage
  \tableofcontents
\newpage

\section{Antenna Design}
\section{PCB Design}
\section{Location Algorithm}
	A first algorithm that can be used to determine the position of the drone is the \textit{Least Mean Square} algorithm. We will elaborate on this first and then compare the performance to the Unscented Kalman Filter, which will be used in the final system. 

	\subsection{Least Mean Square}
	\label{subsec:LMS}

		The LMS algorithm uses the distances between the drone and a number of well-known anchor points to determine the position. The UWB antenna mounted on the drone broadcasts a very short pulse to the anchors, which will return the same pulse when it is received. The board on the drone measures the time between sending the pulse and receiving it back. Because the propagation speed of the pulse through air and the time it takes to process the pulse at the anchors are known, the distance to the anchor points can be determined as follows:

		\begin{equation}
		\centering
			t_{roundtrip} = 2t_{propagation} + t_{process}
		\end{equation} 

		Because the time to process the pulse at the anchor is known, we can calculate the time it takes to bridge the distance between the drone and the anchor point. With c the speed of light ($\SI{3e8}{\meter\per\second}$) this results in an expression for the distance to anchor point i:

		\begin{equation}
		\centering
			d_i = c*t_{propagation}
		\end{equation}

		With $\vec{\textbf{x}}$ denoting the position vector $\begin{bmatrix} x & y & z \end{bmatrix}$ of the drone and $\vec{\textbf{x}_i}$ the position vector $\begin{bmatrix} x_i & y_i & z_i \end{bmatrix}$ of the $i^{th}$ anchor point, this distance can also be expressed as follows:

		\begin{align*}
		\centering
			d_i^2 &= (\vec{\textbf{x}} - \vec{\textbf{x}_i})(\vec{\textbf{x}} - \vec{\textbf{x}_i})^T \\
			&= x^2 - 2x_ix + x_i^2 + y^2 - 2y_iy + y_i^2 + z^2 - 2z_iz + z_i^2
		\end{align*}

		This equation is not linear in x, y and z and therefore it is hard to extract the position vector. With $d_N$ the distance to the $N^{th}$ anchor, the equation can be linearized as follows:

		\begin{align*}
		\centering
			d_i^2 - d_N^2 &= x^2 - 2x_ix + x_i^2 + y^2 - 2y_iy + y_i^2 + z^2 - 2z_iz + z_i^2 - d_N^2 \\
				&= 2x(x_N - x_i) + 2y(y_N - y_i) + 2z(z_N- z_i)+ (x_i^2 - x_N^2) + (y_i^2 - y_N^2)  + (z_i^2 - z_N^2) 
		\end{align*}

		This yields a matrix representation of the form $\textbf{A}\vec{\textbf{x}} = \textbf{B}$ which can be solved for the position vector $\vec{\textbf{x}}$.


	% subsection LMS (end)

\end{document}

